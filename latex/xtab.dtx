% \iffalse meta-comment
%
% xtab.dtx
% Copyright 1998 Peter R. Wilson
%
% This program is provided under the terms of the
% LaTeX Project Public License distributed from CTAN
% archives in directory macros/latex/base/lppl.txt.
%
% Author: Peter Wilson (CUA and NIST)
%         now at: peter.r.wilson@boeing.com
% 
%<*driver>
\documentclass{ltxdoc}
\usepackage{xtab}
\providecommand{\captionsize}{}
\EnableCrossrefs
\CodelineIndex
\setcounter{StandardModuleDepth}{1}
\begin{document}
  \DocInput{xtab.dtx}
\end{document}
%</driver>
% 
% \fi
% \CheckSum{1508}
%
% \DoNotIndex{\',\.,\@M,\@@input,\@addtoreset,\@arabic,\@badmath}
% \DoNotIndex{\@centercr,\@cite}
% \DoNotIndex{\@dotsep,\@empty,\@float,\@gobble,\@gobbletwo,\@ignoretrue}
% \DoNotIndex{\@input,\@ixpt,\@m}
% \DoNotIndex{\@minus,\@mkboth,\@ne,\@nil,\@nomath,\@plus,\@set@topoint}
% \DoNotIndex{\@tempboxa,\@tempcnta,\@tempdima,\@tempdimb}
% \DoNotIndex{\@tempswafalse,\@tempswatrue,\@viipt,\@viiipt,\@vipt}
% \DoNotIndex{\@vpt,\@warning,\@xiipt,\@xipt,\@xivpt,\@xpt,\@xviipt}
% \DoNotIndex{\@xxpt,\@xxvpt,\\,\ ,\addpenalty,\addtolength,\addvspace}
% \DoNotIndex{\advance,\Alph,\alph}
% \DoNotIndex{\arabic,\ast,\begin,\begingroup,\bfseries,\bgroup,\box}
% \DoNotIndex{\bullet}
% \DoNotIndex{\cdot,\cite,\CodelineIndex,\cr,\day,\DeclareOption}
% \DoNotIndex{\def,\DisableCrossrefs,\divide,\DocInput,\documentclass}
% \DoNotIndex{\DoNotIndex,\egroup,\ifdim,\else,\fi,\em,\endtrivlist}
% \DoNotIndex{\EnableCrossrefs,\end,\end@dblfloat,\end@float,\endgroup}
% \DoNotIndex{\endlist,\everycr,\everypar,\ExecuteOptions,\expandafter}
% \DoNotIndex{\fbox}
% \DoNotIndex{\filedate,\filename,\fileversion,\fontsize,\framebox,\gdef}
% \DoNotIndex{\global,\halign,\hangindent,\hbox,\hfil,\hfill,\hrule}
% \DoNotIndex{\hsize,\hskip,\hspace,\hss,\if@tempswa,\ifcase,\or,\fi,\fi}
% \DoNotIndex{\ifhmode,\ifvmode,\ifnum,\iftrue,\ifx,\fi,\fi,\fi,\fi,\fi}
% \DoNotIndex{\input}
% \DoNotIndex{\jobname,\kern,\leavevmode,\let,\leftmark}
% \DoNotIndex{\list,\llap,\long,\m@ne,\m@th,\mark,\markboth,\markright}
% \DoNotIndex{\month,\newcommand,\newcounter,\newenvironment}
% \DoNotIndex{\NeedsTeXFormat,\newdimen}
% \DoNotIndex{\newlength,\newpage,\nobreak,\noindent,\null,\number}
% \DoNotIndex{\numberline,\OldMakeindex,\OnlyDescription,\p@}
% \DoNotIndex{\pagestyle,\par,\paragraph,\paragraphmark,\parfillskip}
% \DoNotIndex{\penalty,\PrintChanges,\PrintIndex,\ProcessOptions}
% \DoNotIndex{\protect,\ProvidesClass,\raggedbottom,\raggedright}
% \DoNotIndex{\refstepcounter,\relax,\renewcommand,\reset@font}
% \DoNotIndex{\rightmargin,\rightmark,\rightskip,\rlap,\rmfamily,\roman}
% \DoNotIndex{\roman,\secdef,\selectfont,\setbox,\setcounter,\setlength}
% \DoNotIndex{\settowidth,\sfcode,\skip,\sloppy,\slshape,\space}
% \DoNotIndex{\symbol,\the,\trivlist,\typeout,\tw@,\undefined,\uppercase}
% \DoNotIndex{\usecounter,\usefont,\usepackage,\vfil,\vfill,\viiipt}
% \DoNotIndex{\viipt,\vipt,\vskip,\vspace}
% \DoNotIndex{\wd,\xiipt,\year,\z@}
%
% \def\fileversion{v2.3}
% \def\filedate{2000/04/09}
% \newcommand*{\Lpack}[1]{\textsf {#1}}           ^^A typeset a package
% \newcommand*{\Lopt}[1]{\textsf {#1}}            ^^A typeset an option
% \newcommand*{\file}[1]{\texttt {#1}}            ^^A typeset a file
% \newcommand*{\Lcount}[1]{\textsl {\small#1}}    ^^A typeset a counter
% \newcommand*{\pstyle}[1]{\textsl {#1}}          ^^A typeset a pagestyle
% \newcommand*{\Lenv}[1]{\texttt {#1}}            ^^A typeset an environment
%
% \title{The \Lpack{xtab} package\thanks{This
%        file has version number \fileversion, last revised
%        \filedate.}}
%
% \author{%
% Peter Wilson\\
% Catholic University of America\thanks{This work was originally
% performed as
% a Guest Researcher at the National Institute of Standards and Technology.} \\
% Now at \texttt{peter.r.wilson@boeing.com}
% }
% \date{\filedate}
% \maketitle
% \begin{abstract}
%  The \Lpack{xtab} package enables long tables to be automatically broken
%  at page boundaries. It is an extension of the \Lpack{supertabular}
%  package and also reduces or eliminates some of its weaknesses.
% \end{abstract}
% \tableofcontents
% \listoftables
%
% \StopEventually{}
%
% 
%
% \section{Introduction}
%
% Although the \Lpack{xtab} package was originally developed as part of a
% suite for typesetting ISO international standards~\cite{PRW96i}, it
% is also applicable for use with the \LaTeX{} standard classes.
% The package is an extension of the \Lpack{supertabular} package
% developed by Johannes Braams and Theo 
% Jurriens.\footnote{\Lpack{supertabular.sty}, version 4.1c, 7 November 1997.}
% It reduces some of the weaknesses noted in the \Lpack{supertabular}
% documentation and provides additional functionality.
%
%
% Section~\ref{sec:usr} provides the user manual for the package which
% enables long tables to be automatically broken across multiple pages.
% Section~\ref{sec:xtab} describes the implementation.
% 
% This manual is typeset according to the conventions of the
% \LaTeX{} \textsc{docstrip} utility which enables the automatic
% extraction of the \LaTeX{} macro source files~\cite{GOOSSENS94}.
%
%
% \section{The \Lpack{xtab} package} \label{sec:usr}
%
%    The \Lpack{supertabular} package provides for the automatic
% breaking of a long table across page boundaries.
% The extension provided here enables the
% heading on the table on the last page to differ from those on earlier
% pages of the table.
% The downside of the extension is that \LaTeX{} has to be run twice
% if the document contains a |supertabular|. However, \LaTeX{} is usually
% run at least twice for any but the simplest document in order to get
% cross-references and Table of Contents, etc., resolved correctly.
%
%    The current version of the extension also either cures or reduces
% following
% weaknesses in the \Lpack{supertabular} package.\footnote{I have
% corresponded with the authors of \Lpack{supertabular} about these.}
% \begin{enumerate}
% \item Sometimes the top caption of a |supertabular| is printed on
%       one page and the body is printed on the following page(s).
%       That is, there is a lonely caption.
% \item Sometimes the last page of a |supertabular| consists of an
%       empty table. That is, just the head and foot of the table are
%       printed.
% \item If the number of lines in the first header for the table differs
%       from the number of lines in subsequent headers, then the
%       continuation pages of the table may be too short or, more
%       troubling, too long.
% \end{enumerate} 
%
% The weaknesses are caused by trying to guess where \TeX{} will put a page 
% break. The package has to guesstimate how long the next entry will be in 
% the table and, if it is too long for the available space, it puts in its
% own page break. If its guess is off too much in one direction, \TeX{} will 
% break the page unexpectadly; if it's off in the other direction
% \Lpack{supertabular} will put in an unnecessary page break.
%
%    The \Lpack{xtab} package has reduced, but perhaps not entirely
% eliminated, these weaknesses. Some hand tuning may still be required.
%
%
%
%    The principal commands available are given in Table~\ref{tab:xtab}.
% \DeleteShortVerb{\|}
% \MakeShortVerb{\?}
% \topcaption{The principal xtab package commands} \label{tab:xtab}
% \tablefirsthead{\hline \multicolumn{1}{|c|}{\textbf{Command}} &
%                        \multicolumn{1}{c|}{\textbf{Effect}} \\ \hline }
% \tablehead{\multicolumn{2}{c}%
%            {{\captionsize\bfseries \tablename\ \thetable{} -- continued from previous page}} \\
%            \hline \multicolumn{1}{|c|}{\textbf{Command}} &
%                   \multicolumn{1}{c|}{\textbf{Effect}} \\ \hline }
% \tablelasthead{\multicolumn{2}{c}%
%            {{\captionsize\bfseries \tablename\ \thetable{} -- concluded from previous page}} \\ 
%            \hline \multicolumn{1}{|c|}{\textbf{Command}} &
%                   \multicolumn{1}{c|}{\textbf{Effect}} \\ \hline }
% \tabletail{\hline \multicolumn{2}{|r|}{{Continued on next page}} \\ \hline}
% \tablelasttail{\hline \hline}
% 
% \begin{center}
% \begin{xtabular}{|l|p{0.5\textwidth}|}
% ?\begin{xtabular}{...}? & This is equivalent to the normal
%                          ?\begin{tabular}{...}? environment.
%                          You supply the specification of the columns
%                          just as for the normal tabular environment.
% \\
% &
%                          All commands that can be used within a 
%                          \Lenv{tabular} 
%                          environment can also be used within
%                          the \Lenv{xtabular} environment.
%  \\
%  &
%     Unlike the \Lenv{tabular} environment which prevents page breaking
% within the tabular, the \Lenv{xtabular} allows page breaking, so that
% tabulars can extend automatically across several pages.
%  \\
%  &
% \Lenv{xtabular}
% starts off with a \Lenv{tabular} environment and checks the amount of
% space left on the page as it adds each row to the tabulation.
% If the space left on the page is too short for another row, then
% it ends the current \Lenv{tabular}, performs a page break and starts
% another \Lenv{tabular} on the following page. This process is repeated
% until all the rows have been output. 
%  \\
%  &
%     There are special commands for captioning an \Lenv{xtabular} as a
% table, and also elements can be automatically inserted after each
% (internal) ?\begin{tabular}? and immediately before each
% ?\end{tabular}?.
%  \\
%  &
%     Do not put a \Lenv{xtabular} in a \Lenv{table} environment, 
% as the \Lenv{table}
% environment keeps its contents on a single page (presumably you
% are using \Lenv{xtabular} because its contents are longer than
% one page).
%  \\
% ?\end{xtabular}? & End the \Lenv{xtabular} environment. \\ \hline
% ?\begin{mpxtabular}?   & Like the \Lenv{xtabular} environment
% except that each `page' is put into a ?minipage? first.
%  \\
%  &
% Thus it is possible to have footnotes inside an \Lenv{mpxtabular}.
% The footnote text is printed at the end of each page.
%  \\
% ?\end{mpxtabular}? & End the \Lenv{mpxtabular} environment. \\ \hline
%                           & \textbf{Note:} If any of the following commands
%                             are used, then they should be placed
%                             before the particular \Lenv{xtabular}
%                             environment that they apply to. \\
% ?\topcaption{...}? & A command to provide a caption for the
%                           table. The caption is placed at the top
%                           of the table. \\
% ?\bottomcaption{...}? & A command to provide a caption for the
%                           table. The caption is placed at the bottom
%                           of the table. \\
% ?\tablecaption{...}? & A command to provide a caption for the
%                           table. The caption is placed at the default
%                           position, which is at the top
%                           of the table.
%  \\
%  & 
%                            \textbf{Notes:} You cannot use the ?\caption?
%                             command but you can put a label after
%                             any of these captioning commands. If you
%                             want captioning, the command must be specified
%                             before the start of the supertabular environment.
%   \\
%   &
%                             The ?\...caption{}? command(s) remain
%                             in effect until changed by another
%                             ?\...caption? command.
%    \\
% ?\tablefirsthead{...}? & Defines the contents of the first occurence
%                          of the tabular head. The tabular head is some
%                          special treatment of the first row in the table.
%                          This command is optional.
%  \\
%  &
%                          If used, the header must be closed by the
%                          end of line command for tabulars (e.g., ?\\?). \\
% ?\tablehead{...}? & Defines the contents of the table head on
%                          subsequent pages.
% \\
% &
%                          For example, you might want to note that
%                          this is a continuation of the table on
%                          the previous page, as well as repeating
%                          any column headings that were given
%                          at the start of the \Lenv{xtabular} by
%                          ?\tablefirsthead?.
%  \\
%  &
%                          The header must be closed like the
%                          ?\tablefirsthead? command.
%  \\
% ?\tablelasthead{...}? & Defines the contents of the table head
%                             on the last page of the table.
%  \\
%  &
%     For example, you might want to note that the table
% is concluded on this page.
%  \\
%  &
%                          The header must be closed like the
%                          ?\tablefirsthead? command.
%  \\
% ?\notablelasthead?  & Switches off the last ?\tablelasthead?.
%                            A ?\tablelasthead? stays in effect until
%                            overwritten by a new ?\tablelasthead?
%                            or cancelled by this command.
%  \\
% ?\tabletail{...}? & The contents of this command are inserted before
%                          the (internal) ?\end{tabular}? on each page except
%                          for the last page of the table.
%  \\
%  &
%                          For example, you might want to note that the
%                          table is continued on the next page. 
%  \\
% ?\tablelasttail{...}? & The contents of this command are inserted before
%                          the final (internal) ?\end{tabular}? of the table.
%  \\
%  &
%                          For example, you might want to note that
%                          this is where the table ends.
%  \\
% \end{xtabular}
% \end{center}
% \DeleteShortVerb{\?}
% \MakeShortVerb{\|}
%
%    As well as the \Lenv{xtabular} and \Lenv{mpxtabular} environments there 
% are the corresponding starred versions 
% (i.e., \Lenv{xtabular*} and \Lenv{mpxtabular*} for use in
% two column mode where the table is meant to span both columns.
%
%  Table~\ref{tab:xtab} was produced by code similar to the following:
% \begin{verbatim}
% \topcaption{The principal xtab package commands} \label{tab:xtab}
% \tablefirsthead{\hline \multicolumn{1}{|c|}{\textbf{Command}} &
%                        \multicolumn{1}{c|}{\textbf{Effect}} \\ \hline }
% \tablehead{\multicolumn{2}{c}%
%            {{\captionsize\bfseries \tablename\ \thetable{} -- 
%              continued from previous page}} \\
%   \hline   \multicolumn{1}{|c|}{\textbf{Command}} &
%            \multicolumn{1}{c|}{\textbf{Effect}} \\ \hline }
% \tablelasthead{\multicolumn{2}{c}%
%            {{\captionsize\bfseries \tablename\ \thetable{} -- 
%              concluded from previous page}} \\
%   \hline   \multicolumn{1}{|c|}{\textbf{Command}} &
%            \multicolumn{1}{c|}{\textbf{Effect}} \\ \hline }
% \tabletail{\hline \multicolumn{2}{|r|}{{Continued on next page}} \\ \hline}
% \tablelasttail{\hline \hline}
% 
% \begin{center}
% \begin{xtabular}{|l|p{0.5\textwidth}|}
% \verb|\begin{xtabular}{...}| & This is equivalent to the normal
%                          \verb|\begin{tabular}{...}| environment.
%                          You supply the specification of the columns
%                          just as for the normal \Lenv{tabular} environment.
%  \\
%  &
%                          All commands that can be used within a \Lenv{tabular}
%                          environment can also be used within
%                          the \Lenv{xtabular} environment.
%  \\
%  &
%     Unlike the \Lenv{tabular} environment which prevents page breaking
% within the tabular, the \Lenv{xtabular} allows page breaking, so that
% tabulars can extend automatically across several pages. 
% ... ... ...
% \verb|\tablelasttail{...}| & The contents of this command are inserted before
%                          the final (internal) \verb|\end{tabular}| of the table.
%  \\
%  &
%     For example, you might want to note that  this is where 
% the table ends.
%  \\
% \end{xtabular}
% \end{center}
% \end{verbatim}
%
%     The table is only broken between rows --- a row will not be split
% across pages. This can lead to some bad page breaks, especially if 
% there are rows with a large vertical height (like some in Table~\ref{tab:xtab}).
% It is best to keep rows not too tall.
%
%    Unkike the \Lenv{table} environment which floats, an \Lenv{xtabular}
% environment is typeset at the point in the document where the environment
% is specified. It is best not to start an \Lenv{xtabular} too close to the
% bottom of a page otherwise there might be an ugly page break.
%
% \DescribeMacro{\shrinkheight}
% The command
% |\shrinkheight{|\meta{length}|}| may be
% used after the first |\\| in the table to modify the allowed
% height of the table on that page. A positive \meta{length} decreases
% the allowed space on the page and a negative \meta{length} increases
% the allowed space.
% 
% For example: \\
% |\shrinkheight{2\baselineskip}| decreases the space per page by 
% two lines. \\
% |\shrinkheight{-\baselineskip}| increases the space per page by
% one line. \\
% Note that I have never tried using this command so I cannot comment 
% on its efficacy. Instead, I use the
% |\xentrystretch| command when necessary.
%
% \DescribeMacro{\xentrystretch}
% The command |\xentrystretch{|\meta{decimal-fraction}|}| can be used before
% a table to modify the amount of vertical space apparently
% consumed by each entry in the subsequent table(s). 
% The default is |\xentrystretch{0.1}| which
% specifies a 10\% overestimate in the vertical space. Similarly,
% |\xentrystretch{0.25}| will overestimate the space by 25\%. A different
% value may be used for each table in order to eliminate, or at least reduce,
% bad page breaks. Increasing the value causes fewer entries to be put on
% a page, thus reducing the chance of \TeX{} putting in a page break before
% the \Lpack{xtab} package is prepared for one.
%
% 
%     You may specify the font used for the 
% |\tablehead| and |\tablelasthead| yourself. 
%
% \textbf{Note:} Within ISO documents, captions shall
% be in bold font. The \file{iso} class also provides 
% a command for
% setting the size of the font used in captions, namely 
% |\captionsize|. The default value for this is set by the \file{iso} class.
% For the curious, the default definition is: \\
% |\newcommand{\captionsize}{\normalsize}|
%
% \subsection{Options}
% 
%     The \Lpack{xtab} package has three options which control the amount of
% information that is written to the \file{.log} file. The
% options are:
% \begin{enumerate}
% \item The option \Lopt{errorshow} (the default) does not
%       write any extra information;
% \item The option \Lopt{pageshow} writes information
%       about when and why \Lpack{xtab} decides to produce a new page;
% \item The option \Lopt{debugshow}, which also includes
%       \Lopt{pageshow}, additionally writes information about each line
%       that is added to the table.
% \end{enumerate}
% 
%     Under normal circumstances \Lpack{xtab} is used without invoking any 
% option. The \Lopt{pageshow} option may be useful when attempting to cure
% a bad page break. The \Lopt{debugshow} option, as its name implies, is
% principally of use to the \Lpack{xtab} developer.
% 
%     Independently of the options, the command 
% |\sstraceon| may be used at any
% point in the document to turn on printing of \Lopt{debugshow} data. This
% can be turned off later by the |\sstraceoff| 
% command, which will stop all \Lopt{\ldots show} printing.
%     
% 
% \section{The implementation} \label{sec:xtab}
%
%    The \Lpack{xtab} package provides an extension to the 
% \Lpack{supertabular} package
% written by  Johannes Braams and Theo 
% Jurriens.\footnote{\texttt{supertabular.sty}, version 4.1c, 7 November 1997.}
% The major portion of the following documentation is taken from
% \file{supertabular.dtx}.
% The package is designed to be used with the \file{iso} class in addition 
% to the usual \file{article}, etc., classes.
%
% \changes{v2}{1997/11/30}{xtab updated from version 3.7 to version 4.1c of supertabular}
% The extension provided here enables the
% heading on the table on the last page to differ from those on earlier
% pages of the table. The implementation of the extension is based
% on ideas in David Carlisle's \Lpack{longtable} package. 
% The downside of the extension is that \LaTeX{} has to be run twice
% if the document contains a |supertabular|. However, \LaTeX{} is usually
% run at least twice for any but the simplest document in order to get
% cross-references and Table of Contents, etc., resolved correctly.
%
%    The current version of the extension also either cures or reduces
% following
% weaknesses in the \Lpack{supertabular} package.\footnote{Only the first
% two of these have been recognised by the authors of \Lpack{supertabular}.}
% \begin{enumerate}
% \item Sometimes the top caption of a |supertabular| is printed on
%       one page and the body is printed on the following page(s).
%       That is, there is a lonely caption.
% \item Sometimes the last page of a |supertabular| consists of an
%       empty table. That is, just the head and foot of the table are
%       printed.
% \item If the number of lines in the first header for the table differs
%       from the number of lines in subsequent headers, then the
%       continuation pages of the table may be too short or, more
%       troubling, too long.
% \end{enumerate} 
%
%    The first version of \Lpack{xtab} imported much of the code from the
% supertabular package (version 3.7) but I found that this did not work 
% well because there were incompatible coded versions of supertabular 
% available on CTAN. Further, I
% found that there were some problems with the original \file{supertabular}
% code in any case.\footnote{I also found a bug in the 4.1b version which
% the authors kindly fixed in version 4.1c.}
% I have to make the assumption that other users may have dissimilar
% or problematic versions, so include all the code here, and thus any errors
% can now be laid at my door.
%
%    The requirement for compatibilty with the \file{iso} class is
% achieved by modifications to the |\ST@caption| command only. Effectively
% this is orthogonal to the code required to implement the extension.
%
%    Now for the code itself. As syntactic sugar, all new macros for the
% extension have the prefix `PWST' to distinguish them from the original
% macros. I have also denoted all extensions to the original 
% \Lpack{supertabular} by introducing them as \emph{Extension:}.
%
%    Announce the name and version of the package, which requires \LaTeXe.
% \changes{v2.2}{1998/10/14}{Rewrote the documentation}
% \changes{v2.2}{1998/10/14}{Changed names from ...supertab... to ...xtab...}
% \changes{v2.2}{1998/10/14}{Kept supertab... as synonyms for xtab..., but deprecated them}
% \changes{v2.2a}{2000/02/15}{Deleted supertab... }
% \changes{v2.2a}{2000/02/15}{Updated to latest ISO class}
% \changes{v2.3}{2000/04/09}{Improved pagebreak control via \cs{xentrystretch} }
%    \begin{macrocode}
%<*xtab>
\NeedsTeXFormat{LaTeX2e}
\ProvidesPackage{xtab}[2000/04/09 v2.3 Extended supertabular package]

%    \end{macrocode}
%
% \begin{macro}{\c@tracingst}
%    There are three options with the package which control the amount
%    of information written to the log file:
%    \begin{enumerate}
%    \item \Lopt{errorshow} (the default) no extra information
%    \item \Lopt{pageshow} writes information about page breaking
%    \item \Lopt{debugshow} adds information about each line that
%          is added to the tabular
%    \end{enumerate}
% \changes{v2}{1997/11/30}{Added options to xtab}
%    \begin{macrocode}
\newcount\c@tracingst
\DeclareOption{errorshow}{\c@tracingst\z@}
%    \end{macrocode}
% \emph{Extension:} The next line in the original code did not do what
% the authors intended; the number should have been 3 rather than 2.
%    \begin{macrocode}
%%%%\DeclareOption{pageshow}{\c@tracingst\tw@}
\DeclareOption{pageshow}{\c@tracingst\thr@@}
\DeclareOption{debugshow}{\c@tracingst5\relax}
\ProcessOptions

%    \end{macrocode}
% \end{macro}   %%%%%%%%%%%%%%%%%%%%%%%%%%%%%%%%%%%
%
% \begin{macro}{\if@topcaption}
% \begin{macro}{\topcaption}
% \begin{macro}{\bottomcaption}
%    The user-commands |\topcaption| and |\bottomcaption| set the
%    flag |@topcaption| to determine where to put the
%    tablecaption. The default is to put the caption on the top of
%    the table 
%    \begin{macrocode}
\newif\if@topcaption \@topcaptiontrue
\def\topcaption{\@topcaptiontrue\tablecaption}
\def\bottomcaption{\@topcaptionfalse\tablecaption}

%    \end{macrocode}
% \end{macro}
% \end{macro}
% \end{macro}
%
% \begin{macro}{\PWST@thecaption}
% \begin{macro}{\PWSTcapht}
%  \emph{Extension:} |\PWST@thecaption| is used to store
% the text of the table's caption. The vertical space required by
% a caption is stored in |\PWSTcapht|.
% \changes{v2}{1997/11/30}{Added PWST@thecaption and PWSTcapht}
%    \begin{macrocode}
\gdef\PWST@thecaption{}
\newdimen\PWSTcapht
%    \end{macrocode}
% \end{macro}
% \end{macro}
%
% \begin{macro}{\tablecaption}
% \begin{macro}{\@xtablecaption}
%    This command has to function exactly like |\caption| does except
%    it has to store its argument (and the optional argument) for
%    later processing \emph{within} the supertabular environment. 
%
%    \begin{macrocode}
\long\def\tablecaption{%
  \refstepcounter{table}\@dblarg{\@xtablecaption}}
\long\def\@xtablecaption[#1]#2{%
%    \end{macrocode}
% \emph{Extension:} I store the caption text for later measurement.
%    \begin{macrocode}
  \long\gdef\PWST@thecaption{#2}%
%    \end{macrocode}
% Finish up with the original code.
%    \begin{macrocode}
  \long\gdef\@process@tablecaption{\ST@caption{table}[#1]{#2}}}
\global\let\@process@tablecaption\relax

%    \end{macrocode}
% \end{macro}
% \end{macro}
%
%  \begin{macro}{\ifST@star}
%    This switch is used in the internal macros to remember which
%    kind of environment was started.
%    \begin{macrocode}
\newif\ifST@star
%    \end{macrocode}
%  \end{macro}
%
% \begin{macro}{\ifST@mp}
%    This flag is used in the internal macros to remember if the
%    tabular is to be put in a minipage.
% \changes{v2}{1997/11/30}{Added ifST@mp}
%    \begin{macrocode}
\newif\ifST@mp
%    \end{macrocode}
% \end{macro}
%
%  \begin{macro}{\ST@wd}
%    For the \Lenv{supertabular*} environment it is necessary to
%    store the intended width of the tabular.
%    \begin{macrocode}
\newdimen\ST@wd
%    \end{macrocode}
%  \end{macro}
%
% \begin{macro}{\ST@rightskip}
% \begin{macro}{\ST@leftskip}
% \begin{macro}{\ST@parfillskip}
%    For the \Lenv{mpsupertabular} environments we need special versions
%    of |\leftskip|, |\rightskip| and |\parfillskip|.
% \changes{v2}{1997/11/30}{Added ST@rightskip, ST@leftskip and ST@parfillskip}
%    \begin{macrocode}
\newskip\ST@rightskip
\newskip\ST@leftskip
\newskip\ST@parfillskip

%    \end{macrocode}
% \end{macro}
% \end{macro}
% \end{macro}
%
% \begin{macro}{\@initisotab}
%    Required for ISO class, and check if class loaded.
% \changes{v2.2a}{2000/02/15}{Added \cs{@initisotab} command}
%    \begin{macrocode}
\@ifundefined{@initisotab}{%
  \newcommand{\@initisotab}{}
  \newif\ifinfloat}{\typeout{xtab using iso captions}}

%    \end{macrocode}
% \end{macro}
%
%  \begin{macro}{\ST@caption}
%    This is a redefinition of LaTeX's |\@caption|, |\@makecaption| is
%    called within a group so as not to return to |\normalsize|
%    globally. In the original a fix was made for the `feature' of the
%    |\@makecaption| of \file{article.sty} and friends that a caption 
%    \emph{always} gets a |\vskip 10pt| at the top and \emph{none}
%    at the bottom; if a user
%    wants to precede his table with a caption this results in a
%    collision. This fix is not implemented here as I think it should
% be done by the user modifying |\beforecaptionskip| and |\aftercaptionskip|.
%
% \emph{Extension:} The ISO captioning is also initialised.
%
%    \begin{macrocode}
\long\def\ST@caption#1[#2]#3{\par%
  \@initisotab
  \addcontentsline{\csname ext@#1\endcsname}{#1}%
                  {\protect\numberline{%
                    \csname the#1\endcsname}{\ignorespaces #2}}
  \begingroup
    \@parboxrestore
    \normalsize
  %%  \if@topcaption \vskip -10\p@ \fi
    \@makecaption{\csname fnum@#1\endcsname}{\ignorespaces #3}\par
  %%  \if@topcaption \vskip 10\p@ \fi
  \endgroup
%    \end{macrocode}
% \emph{Extension:} The height of the caption is subtracted from the available
% space on the page.
% \changes{v2.3}{2000/04/09}{Subtracted caption height from available space in
% \cs{ST@caption} }
%    \begin{macrocode}
  \global\advance\ST@pageleft -\PWSTcapht
  \ST@trace\tw@{Added caption. Space left for xtabular: \the\ST@pageleft}
}

%    \end{macrocode}
% \end{macro}
%
% \begin{macro}{\tablehead}
% \begin{macro}{\tablefirsthead}
%
%    |\tablehead| activates the new tabular |\cr| commands.
% \changes{v2}{1997/11/30}{Using version 4.1b of tablehead}
%    \begin{macrocode}
\newcommand\tablehead[1]{%
  \gdef\@tablehead{%
  \noalign{%
    \global\let\@savcr=\\
    \global\let\\=\org@tabularcr}%
  #1%
  \noalign{\global\let\\=\@savcr}}}
\tablehead{}
\newcommand\tablefirsthead[1]{\gdef\@table@first@head{#1}}

%    \end{macrocode}
% \end{macro}
% \end{macro}
%
% \begin{macro}{\c@PWSTtable}
% \begin{macro}{\PWSTlastpage}
% \begin{macro}{\PWSTcurpage}
% \begin{macro}{\PWSTpenultimate}
% \begin{macro}{\PWSTtempc}
% \begin{macro}{\PWSTlines}
% \begin{macro}{\PWSThead}
% \begin{macro}{\PWSTlasthead}
% \emph{Extension:} These are counters for the supertabular extension. 
% \Lcount{c@PWSTtable}
% counts the number of supertabulars in case one or more are not captioned.
% \Lcount{PWSTlastpage} is a counter holding the number of pages that a
% supertabular uses and \Lcount{PWSTpenultimate} is the penultimate page.
% \Lcount{PWSTcurpage} counts the current number
% of supertabular pages processed. \Lcount{PWSTtempc} is a scratch
% counter for page processing.
% \changes{v2}{1997/11/30}{Changed PWSToldpages to PWSTlastpage}
% \changes{v2}{1997/11/30}{Added counters PWSTpenultimate, PWSTlines, PWSThead and PWSTlasthead}
%    \begin{macrocode}
\newcounter{PWSTtable}
\newcount\PWSTlastpage
\newcount\PWSTpenultimate
\newcount\PWSTcurpage
\newcount\PWSTtempc
%    \end{macrocode}
% \emph{Extension:} \Lcount{PWSTlines} is used to count the number of 
% supertabular entry
% lines on a page. Estimates of the number of lines in the normal table
% heading is held by \Lcount{PWSThead}, and similarly \Lcount{PWSTlasthead}
% is for the number of lines in the last heading.
%    \begin{macrocode}
\newcount\PWSTlines
%%% \newcount\PWSThead
%%% \newcount\PWSTlasthead
%    \end{macrocode}
% \end{macro}
% \end{macro}
% \end{macro}
% \end{macro}
% \end{macro}
% \end{macro}
% \end{macro}
% \end{macro}
%
% \begin{macro}{\iffirstcall}
% \emph{Extension:} This is used by the extension code to flag if the presumed last 
% page overflows. If overflow occurs, then |firstcall| is set to false.
% \changes{v2}{1997/11/30}{Added iffirstcall}
%    \begin{macrocode}
\newif\iffirstcall
%    \end{macrocode}
% \end{macro}
%
% \begin{macro}{\PWST@lastht}
% \begin{macro}{\PWST@generalht}
% \begin{macro}{\PWST@ht}
% \emph{Extension:} The estimated height of a table header and tail (i.e., the height of
% an empty table) for the last page of a supertabular is stored in
% |\PWSTlastht|. Similarly, the corresponding height of an empty table on
% a general page (neither the first nor the last) is stored in 
% |\PWSTgeneralht|. |\PWST@ht| is a scratch variable.
% \changes{v2}{1997/11/30}{Added PWST@generalht, PWST@lastht and PWST@ht}
%    \begin{macrocode}
\newdimen\PWST@lastht
\newdimen\PWST@generalht
\newdimen\PWST@ht

%    \end{macrocode}
% \end{macro}
% \end{macro}
% \end{macro}
%
% \begin{macro}{\tablelasthead}
% \begin{macro}{\@table@last@head}
% \begin{macro}{\notablelasthead}
% \emph{Extension:} |\tablelasthead| is the extension user command 
% to specify the heading for 
% the last page of a supertabular. The command |\notablelasthead|
% switches off the last heading. This has to be used if a last headed
% table precedes one that does not have a special last head.
%    \begin{macrocode}
\newcommand{\tablelasthead}[1]{\gdef\@table@last@head{#1}}
\newcommand{\notablelasthead}{\let\@table@last@head\relax}
%    \end{macrocode}
%    Now initialize these commands.
%    \begin{macrocode}
\tablelasthead{}
\notablelasthead
%    \end{macrocode}
% \end{macro}
% \end{macro}
% \end{macro}
%
% \begin{macro}{\tabletail}
% \begin{macro}{\tablelasttail}
%    |\tabletail| is the user command to specify the appearance of
% the bottom of each tabular on a page. Special treatment is
% given to the end of the supertabular via the |\tablelasttail|
% command.
%
%    If the user uses an extra amount of tabular-data (like
%    |\multicolumn|) in |\tabletail| \TeX\ starts looping because of
%    the definition of |\nextline|. So make |\\| act like just a |\cr|
%    inside this tail to prevent the loop.  Save and restore the value
%    of |\\|
%
% \changes{v2}{1997/11/30}{tabletail revised to 4.1b version}
% \changes{v2.1}{1998/4/13}{Added initialization of tablelasttail}
%    \begin{macrocode}
\newcommand\tabletail[1]{%
  \gdef\@tabletail{%
    \noalign{%
      \global\let\@savcr=\\
      \global\let\\=\org@tabularcr}%
    #1%
    \noalign{\global\let\\=\@savcr}}}
\tabletail{}
\newcommand\tablelasttail[1]{\gdef\@table@last@tail{#1}}
\tablelasttail{}

%    \end{macrocode}
% \end{macro}
% \end{macro}
%
%  \begin{macro}{\sttraceon}
%  \begin{macro}{\sttraceoff}
%    The original supertabular included a tracing mechanism to 
% follow the decisions supertabular made about page breaking.
% This is now also used as a debugging mechanism for the
% extension.
% 
% \changes{v2}{1997/11/30}{Tracing revised}
%    \begin{macrocode}
\newcommand\sttraceon{\c@tracingst5\relax}
\newcommand\sttraceoff{\c@tracingst\z@}
%    \end{macrocode}
%  \end{macro}
%  \end{macro}
%
%  \begin{macro}{\ST@trace}
%    A macro that gets the trace message as its argument
% \changes{v2}{1997/11/30}{ST@trace revised to version 4.1b}
%    \begin{macrocode}
\newcommand\ST@trace[2]{%
  \ifnum\c@tracingst>#1\relax
    \GenericWarning
      {(xtab)\@spaces\@spaces}
      {Package xtab: #2}%
  \fi
}

%    \end{macrocode}
%  \end{macro}
%
%  \begin{macro}{\ST@pageleft}
%    This register holds the estimate of the amount of space left over
%    on the current page. This is used in the decision when to start a
%    new page.
%    \begin{macrocode}
\newdimen\ST@pageleft
%    \end{macrocode}
%  \end{macro}
%
% \begin{macro}{\shrinkheight}
% \begin{macro}{\setSTheight}
%    |\shrinkheight| is a command to diminish the value of 
%    |\ST@pageleft| if necessary.
%
%    |\setSTheight| sets the value of |\ST@pageleft| if necessary.
%
% \changes{v2}{1997/30/11}{Added shrinkheight and setSTheight commands}
%    \begin{macrocode}
\newcommand*\shrinkheight[1]{%
  \noalign{\global\advance\ST@pageleft-#1\relax}}
\newcommand*\setSTheight[1]{%
  \noalign{\global\ST@pageleft=#1\relax}}
%    \end{macrocode}
% \end{macro}
% \end{macro}
%
% \begin{macro}{\xentrystretch}
% \begin{macro}{\PWST@xentrystretch}
% \emph{Extension:} Provide a user and internal command for fudging
% the estimated space taken by a table entry. Initialise to 10\% increase.
% \changes{v2.3}{2000/04/09}{Added \cs{xentrystretch} command}
%    \begin{macrocode}
\newcommand{\xentrystretch}[1]{\def\PWST@xentrystretch{#1}}
\xentrystretch{0.1}

%    \end{macrocode}
% \end{macro}
% \end{macro}
%    
%
% \begin{macro}{\ST@headht}
% \begin{macro}{\ST@tailht}
%    The register |\ST@headht| holds the height of the first head of
%    a \Lenv{supertabular}. The register |\ST@tailht| holds the
%    height of the tail.
%
% \changes{v2}{1997/11/30}{ST@heightht and ST@tailht defined}
%    \begin{macrocode}
\newdimen\ST@headht
\newdimen\ST@tailht
%    \end{macrocode}
% \end{macro}
% \end{macro}
%
% \begin{macro}{\ST@pagesofar}
%    Register |\ST@pagesofar| stores the estimate of the amount of the
%    page already filled up.
%    \begin{macrocode}
\newdimen\ST@pagesofar
%    \end{macrocode}
% \end{macro}
%
% \begin{macro}{\ST@pboxht}
%    The measured (total) height of a parbox argument.
%    \begin{macrocode}
\newdimen\ST@pboxht
%    \end{macrocode}
% \end{macro}
%
% \begin{macro}{\ST@lineht}
% \begin{macro}{\ST@stretchht}
% \begin{macro}{\ST@prevht}
%    The estimated height of a normal line is stored in |\ST@lineht|.
%    The register |\ST@stretchht| is used to store the difference
%    between the normal line height and the line height when |\arraystretch|
%    has a non-standard value. This is used in the case when p-box
%    entries are added to the tabular. |\ST@prevht| stores the height
%    of the previous line to use it as an estimate for the height
%    of the next line. This is needed for a better estimate of when to
%    break the tabular.
%
%    \begin{macrocode}
\newdimen\ST@lineht
\newdimen\ST@stretchht
\newdimen\ST@prevht
%    \end{macrocode}
% \end{macro}
% \end{macro}
% \end{macro}
%
% \begin{macro}{\ST@toadd}
%    When a tabular row is ended with |\\[...]| we need to temporarily
%    store the optional argument in |\ST@toadd|.
%    \begin{macrocode}
\newdimen\ST@toadd
%    \end{macrocode}
% \end{macro}
%
% \begin{macro}{\ST@dimen}
%    A private scratch dimension register.
%    \begin{macrocode}
\newdimen\ST@dimen
%    \end{macrocode}
% \end{macro}
%
% \begin{macro}{\ST@pbox}
%    A box register to store the contents of a parbox.
%    \begin{macrocode}
\newbox\ST@pbox

%    \end{macrocode}
% \end{macro}
%
%
%  \begin{macro}{\ST@tabularcr}
%  \begin{macro}{\ST@xtabularcr}
% \begin{macro}{\ST@argtabularcr}
%    These are redefinitions of |\@tabularcr| and |\@xtabularcr|. This
%    is needed to include |\ST@cr| in the definition of
%    |\@xtabularcr|. 
%
%    All redefined macros have names that are similar to the original
%    names, except with a leading `ST'.
%    \begin{macrocode}
\def\ST@tabularcr{%
  {\ifnum0=`}\fi
  \@ifstar{\ST@xtabularcr}{\ST@xtabularcr}}
\def\ST@xtabularcr{%
  \@ifnextchar[%]
    {\ST@argtabularcr}%
    {\ifnum0=`{\fi}\cr\ST@cr}}
\def\ST@argtabularcr[#1]{%
  \ifnum0=`{\fi}%
  \ifdim #1>\z@
    \unskip\ST@xargarraycr{#1}
  \else
    \ST@yargarraycr{#1}%
  \fi}
%    \end{macrocode}
%  \end{macro}
%  \end{macro}
%  \end{macro}
%
%
% \begin{macro}{\ST@xargarraycr}
% \begin{macro}{\ST@yargarraycr}
%    In this case we need to copy the value of the optional argument
%    of |\\| in our private register |\ST@toadd|.
%    \begin{macrocode}
\def\ST@xargarraycr#1{%
  \@tempdima #1\advance\@tempdima \dp \@arstrutbox
  \vrule \@height\z@ \@depth\@tempdima \@width\z@ \cr
  \noalign{\global\ST@toadd=#1}\ST@cr}
%    \end{macrocode}
%
%    Here we need to insert |\ST@cr|
%    \begin{macrocode}
\def\ST@yargarraycr#1{%
  \cr\noalign{\vskip #1\global\ST@toadd=#1}\ST@cr}

%    \end{macrocode}
% \end{macro}
% \end{macro}
%
% \begin{macro}{\ST@startpbox}
%    The macros that deal with parbox columns need to be redefined,
%    because we need to know the size of the parbox. 
% \changes{v2}{1997/11/30}{ST@startpbox changed to 4.1b version}
%    \begin{macrocode}
\def\ST@startpbox#1{%
%    \end{macrocode}
%    To achieve our goal we need to save the text in box.
%    \begin{macrocode}
  \setbox\ST@pbox\vtop\bgroup\hsize#1\@arrayparboxrestore}
%    \end{macrocode}
% \end{macro}
%
% \begin{macro}{\ST@astartpbox}
%    supertabular version of |\@astartpbox|.
%    \begin{macrocode}
\def\ST@astartpbox#1{%
  \bgroup\hsize#1%
  \setbox\ST@pbox\vtop\bgroup\hsize#1\@arrayparboxrestore}
%    \end{macrocode}
% \end{macro}
%
% \begin{macro}{\ST@endpbox}
% \begin{macro}{\ST@aendpbox}
%    supertabular versions of |\@endpbox| and |\@aendpbox|.
%    \begin{macrocode}
\def\ST@endpbox{%
  \@finalstrut\@arstrutbox\par\egroup
  \ST@dimen=\ht\ST@pbox
  \advance\ST@dimen by \dp\ST@pbox
  \ifnum\ST@pboxht<\ST@dimen
    \global\ST@pboxht=\ST@dimen
  \fi
  \ST@dimen=\z@
  \box\ST@pbox\hfil}
\def\ST@aendpbox{%
  \@finalstrut\@arstrutbox\par\egroup
  \ST@dimen=\ht\ST@pbox
  \advance\ST@dimen by \dp\ST@pbox
  \ifnum\ST@pboxht<\ST@dimen
    \global\ST@pboxht=\ST@dimen
  \fi
  \ST@dimen=\z@
  \unvbox\ST@pbox\egroup\hfil}

%    \end{macrocode}
% \end{macro}
% \end{macro}
%
%
% \begin{macro}{\estimate@lineht}
%    Estimates the height of normal line taking |\arraystretch| into
%    account. Also computes the difference between a `normal' line
%    and a stretched one.
% \changes{v2.3}{2000/04/09}{Added fudge to \cs{estimate@lineht} }
%    \begin{macrocode}
\def\estimate@lineht{%
  \ST@lineht=\arraystretch \baselineskip
  \global\advance\ST@lineht by 1\p@
  \ST@stretchht\ST@lineht\advance\ST@stretchht-\baselineskip
  \ifdim\ST@stretchht<\z@\ST@stretchht\z@\fi
  \ST@trace\tw@{Basic line height: \the\ST@lineht\MessageBreak%
                Arrayed line height: \the\ST@stretchht}%
  \global\advance\ST@lineht \PWST@xentrystretch\ST@lineht
  \ST@trace\tw@{Stretched line height: \the\ST@lineht}%
}

%    \end{macrocode}
% \end{macro}
%
% \begin{macro}{\@calfirstpageht}
%    Estimates the space left on the current page and decides whether
%    the tabular can be started on this page or on a new page. Aspects
%    of the original code are modified for the extension.
%    \begin{macrocode}
\def\@calfirstpageht{%
  \ST@trace\tw@{Calculating height of xtabular on first page}%
%    \end{macrocode}
%    The \TeX{} register |\pagetotal| contains the height of the page
%    sofar, the \LaTeX{} register |\@colroom| contains the height of
%    the column.
%    \begin{macrocode}
  \global\ST@pagesofar\pagetotal
  \global\ST@pageleft\@colroom
  \ST@trace\tw@{Height of previous text = \the\pagetotal; \MessageBreak
                Height of column = \the\ST@pageleft}%
%    \end{macrocode}
%    When we are in twocolumn mode \TeX{} may still be collecting
%    material for the first column although there seems to be no space
%    left. In this case we have to check against two times |\ST@pageleft|.
%    \begin{macrocode}
  \if@twocolumn
    \ST@trace\tw@{two column mode}%
    \if@firstcolumn
      \ST@trace\tw@{First column}%
      \ifnum\ST@pagesofar > \ST@pageleft 
        \global\ST@pageleft=2\ST@pageleft
        \ifnum\ST@pagesofar > \ST@pageleft
          \newpage\@calnextpageht
          \ST@trace\tw@{starting new page}%
        \else
%    \end{macrocode}
%    In this case we're in the second column, so we have to compensate
%    for the material in the first column.
%    \begin{macrocode}
          \ST@trace\tw@{Second column}%
          \global\advance\ST@pageleft -\ST@pagesofar
          \global\advance\ST@pageleft -\@colroom
        \fi
%    \end{macrocode}
%    When |\ST@pagesofar| is smaller than |\ST@pageleft| \TeX{} is still
%    collecting material for the first column, so we can start a new
%    tabular environment like we do on a single column page.
%    \begin{macrocode}
     \else
       \global\advance\ST@pageleft by -\ST@pagesofar
       \global\ST@pagesofar\z@
      \fi
    \else
%    \end{macrocode}
%    When we end up here, \TeX{} has already decided it had enough 
%    material for the first column and is building the second column.
%    \begin{macrocode}
      \ST@trace\tw@{Second column}%
      \ifnum\ST@pagesofar > \ST@pageleft
        \ST@trace\tw@{starting new page}%
        \newpage\@calnextpageht
      \else
        \global\advance\ST@pageleft by -\ST@pagesofar
        \global\ST@pagesofar\z@
      \fi
    \fi
  \else
%    \end{macrocode}
%    In one column mode there is a simple decision.
%    \begin{macrocode}
     \ST@trace\tw@{one column mode}%
     \ifnum\ST@pagesofar > \ST@pageleft
       \ST@trace\tw@{starting new page}%
     \newpage\@calnextpageht
%    \end{macrocode}
%    When we are not starting a new page subtract the size of the
%    material already on it from the available space.
%    \begin{macrocode}
    \else
      \global\advance\ST@pageleft by -\ST@pagesofar
      \global\ST@pagesofar\z@
    \fi
  \fi
  \ST@trace\tw@{Available height: \the\ST@pageleft}%
%    \end{macrocode}
%    Now we need to know the height of the head of the table. In order
%    to measure this we typeset it in a normal \Lenv{tabular} environment.
%    \begin{macrocode}
  \ifx\@@tablehead\@empty
    \ST@headht=\z@
  \else
    \setbox\@tempboxa=\vbox{\@arrayparboxrestore
      \ST@restore
      \expandafter\tabular\expandafter{\ST@tableformat}%
      \@@tablehead\endtabular}%
    \ST@headht=\ht\@tempboxa\advance\ST@headht\dp\@tempboxa
  \fi
  \ST@trace\tw@{Height of head: \the\ST@headht}%
%    \end{macrocode}
%    To decide when to start a new page, we need to know the vertical
%    size of the tail of the table.
%    \begin{macrocode}
  \ifx\@tabletail\@empty
    \ST@tailht=\z@
  \else
    \setbox\@tempboxa=\vbox{\@arrayparboxrestore
      \ST@restore
      \expandafter\tabular\expandafter{\ST@tableformat}
        \@tabletail\endtabular}
    \ST@tailht=\ht\@tempboxa\advance\ST@tailht\dp\@tempboxa
  \fi
%    \end{macrocode}
%    We add the average height of a line to this because when we
%    decide to continue the tabular we need to have enough space left
%    for one line and the tail.
%    \begin{macrocode}
  \advance\ST@tailht by \ST@lineht
  \ST@trace\tw@{Height of tail: \the\ST@tailht}%
%    \end{macrocode}
%    \begin{macrocode}
  \ST@trace\tw@{Maximum space for xtabular: \the\ST@pageleft}%
%    \end{macrocode}
%    Now we decide whether we can continue on the current page or
%    whether we need to start a new page. We assume that the minimum
%    height of a tabular is the height of the head and tail and one line
%    of data. If that doesn't fit, start a new page.
%    \begin{macrocode}
  \@tempdima\ST@headht
  \advance\@tempdima\ST@lineht
  \advance\@tempdima\ST@tailht
%    \end{macrocode}
% \emph{Extension:} I also add the height of the caption to the required
% space. The amount to be added depends on whether it is a top or bottom
% caption. Allowance is also made for skips around the caption.
%    \begin{macrocode}
  \if@topcaption
    \setbox\@tempboxa=\vbox{\PWST@thecaption}
    \PWSTcapht=\ht\@tempboxa\advance\PWSTcapht\dp\@tempboxa
    \advance\PWSTcapht by 20\p@
  \else
    \PWSTcapht = 10\p@
  \fi
  \ST@trace\tw@{Caption height: \the\PWSTcapht}%
  \advance\@tempdima\PWSTcapht
%    \end{macrocode}
% Continue with the original code.
%    \begin{macrocode}
  \ST@trace\tw@{Minimum height of xtabular: \the\@tempdima}%
  \ifnum\@tempdima>\ST@pageleft
    \ST@trace\tw@{starting new page}%
%    \end{macrocode}
% \emph{Extension:} The next line in the original code is 
% |\newpage\@calnextpageht|. I need to start a new page, making allowance
% for the space required by the caption.
%    \begin{macrocode}
    \newpage
    \global\ST@pageleft\@colroom
    \global\advance\ST@pageleft by -\PWSTcapht
    \global\ST@pagesofar=\z@
%    \end{macrocode}
% Finish up with the original code.
%    \begin{macrocode}
  \fi
}  % end \@calfirstpageheight

%    \end{macrocode}
% \end{macro}
%
% \begin{macro}{\@calnextpageht}
%    This calculates the maximum height of the tabular on all
%    subsequent pages of the supertabular environment.
%    \begin{macrocode}
\def\@calnextpageht{%
  \ST@trace\tw@{Calculating height of xtabular on next page}%
  \global\ST@pageleft\@colroom
  \global\ST@pagesofar=\z@
  \ST@trace\tw@{Maximum space for xtabular: \the\ST@pageleft}%
  }

%    \end{macrocode}
% \end{macro}
%
% \begin{macro}{\PWSTcalchtlines}
% \emph{Extension:} A macro to calculate the space required by an empty table
% and the number of lines in an empty table. 
% \changes{v2.3}{2000/04/09}{Don't calculate number of head/tail lines}
%
%    The appropriate heads and tails are typeset in a temporary box
% so we can measure them.
% \changes{v2}{1997/11/30}{Added PWSTcalchtlines routine}
%    \begin{macrocode}
\newcommand{\PWSTcalchtlines}{%
%    \end{macrocode}
%    Measure the lasttail.
%    \begin{macrocode}
  \setbox\@tempboxa=\vbox{\@arrayparboxrestore
    \ST@restore
    \expandafter\tabular\expandafter{\ST@tableformat}%
    \@table@last@tail\endtabular}%
  \PWST@ht=\ht\@tempboxa\advance\PWST@ht\dp\@tempboxa
  \global\PWST@lastht = \PWST@ht
%    \end{macrocode}
%    And repeat for the lasthead.
%    \begin{macrocode}
  \setbox\@tempboxa=\vbox{\@arrayparboxrestore
    \ST@restore
    \expandafter\tabular\expandafter{\ST@tableformat}%
    \@table@last@head\endtabular}%
  \PWST@ht = \ht\@tempboxa\advance\PWST@ht\dp\@tempboxa
  \global\advance\PWST@lastht by \PWST@ht
  \ST@trace\tw@{Height of empty xtabular on last page = \the\PWST@lastht}%
%    \end{macrocode}
%    Now repeat pretty well all of the above for a general table (i.e.,
% one that is not on the first page nor the designated last page).
%
%    First the tail.
%    \begin{macrocode}
  \setbox\@tempboxa=\vbox{\@arrayparboxrestore
    \ST@restore
    \expandafter\tabular\expandafter{\ST@tableformat}%
    \@tabletail\endtabular}%
  \PWST@ht=\ht\@tempboxa\advance\PWST@ht\dp\@tempboxa
  \global\PWST@generalht = \PWST@ht
%    \end{macrocode}
%    And on to the general head.
%    \begin{macrocode}
  \setbox\@tempboxa=\vbox{\@arrayparboxrestore
    \ST@restore
    \expandafter\tabular\expandafter{\ST@tableformat}%
    \@tablehead\endtabular}%
  \PWST@ht = \ht\@tempboxa\advance\PWST@ht\dp\@tempboxa
  \global\advance\PWST@generalht by \PWST@ht  
} % end \PWSTcalchtlines

%    \end{macrocode}
% \end{macro}
%
% \begin{macro}{\PWSTcalnextpageht}
% \emph{Extension:} From some experiments that I ran it appeared as 
% though the \Lpack{supertabular} package ignored the 
% possibility that the
% space required for the table header and tail on pages after the
% first one might be different. If the subsequent head/tail combination
% were longer (i.e., took more vertical space) then the table could
% overflow the page. This is an attempt to fix this problem by
% calculating the actual minimum space required after the first page.
%
%    The calculations are similar to, but simpler, than those for
% |\@calfirstpageht|.
% \changes{v2}{1997/11/30}{Added PWSTcalnextpageht routine}
%    \begin{macrocode}
\newcommand{\PWSTcalnextpageht}{%
  \ifnum\PWSTcurpage = \PWSTpenultimate
    \ST@trace\tw@{Calculating height of xtabular on last page}%
%    \end{macrocode}
%    We are on the penultimate page, so get the height of the 
% last head/tail.
%    \begin{macrocode}
    \PWST@ht=\PWST@lastht
%    \end{macrocode}
%    Otherwise I need the general page data.
%    \begin{macrocode}
  \else
    \ST@trace\tw@{Calculating height of xtabular on next general page}%
    \PWST@ht=\PWST@generalht
  \fi
%    \end{macrocode}
%    Having dealt with the two cases, I can now calculate the minimum
% space for a supertabular on the following page.
%    \begin{macrocode}
  \global\ST@pageleft\@colroom
  \global\advance\ST@pageleft -\PWST@ht
  \global\ST@pagesofar=\z@
  \ST@trace\tw@{Available space for xtabular: \the\ST@pageleft}%
}

%    \end{macrocode}
% \end{macro}
%
% \begin{macro}{\x@supertabular}
%    The various supertabular environments share a lot of code. Thus,
%    to avoid needless repetition, the shared code is defined
%    in this macro.
%
%    This macro has been modified as part of the supertabular extension.
%    \begin{macrocode}
\def\x@supertabular{%
%    \end{macrocode}
%    First save the original definition of |\tabular| and then make it
%    point to |\inner@tabular|. This is done to enable supertabular
%    cells to contain a \Lenv{tabular} environment without getting
%    unexpected results when the \Lenv{supertabular} would be split
%    across this inner \Lenv{tabular} environment.
%    \begin{macrocode}
  \let\org@tabular\tabular
  \let\tabular\inner@tabular
%    \end{macrocode}
%    The same has to be done for the \Lenv{tabular*} environment. The
%    coding is more verbose.
%    \begin{macrocode}
  \expandafter\let
    \csname org@tabular*\expandafter\endcsname
    \csname tabular*\endcsname
  \expandafter\let\csname tabular*\expandafter\endcsname
    \csname inner@tabular*\endcsname
%    \end{macrocode}
% \emph{Extension:} The original code printed out the top caption
% at this point. If there is too little space on the first page of
% the table, the tabular data is printed on the following page.
% If this is the case (and its not known yet whether it is), then
% the caption should also be printed on the following page.
%    \begin{macrocode}
%%%%  \if@topcaption \@process@tablecaption \fi
%    \end{macrocode}
% Back to the original code. Save the original definition of |\\|.
%    \begin{macrocode}
  \global\let\@oldcr=\\
%    \end{macrocode}
%    Save the current value of |\baselineskip|, as we need it in the
%    calculation of the average height of a line. 
%    \begin{macrocode}
  \def\baslineskp{\baselineskip}%
%    \end{macrocode}
%    We have to check whether \texttt{array.sty} was loaded, because
%    some of the internal macros have different names.
%    \begin{macrocode}
  \ifx\undefined\@classix
%    \end{macrocode}
%    Save old |\@tabularcr| and insert the definition of
%    |\@stabularcr|. 
%    \begin{macrocode}
    \let\org@tabularcr\@tabularcr
    \let\@tabularcr\ST@tabularcr
%    \end{macrocode}
%    Activate the new parbox algorithm.
%    \begin{macrocode}
    \let\org@startpbox=\@startpbox
    \let\org@endpbox=\@endpbox
    \let\@@startpbox=\ST@startpbox
    \let\@@endpbox=\ST@endpbox
  \else
%    \end{macrocode}
%    When \texttt{array.sty} was loaded things are a bit different.
%    \begin{macrocode}
    \let\org@tabularcr\@arraycr
    \let\@arraycr\ST@tabularcr
    \let\org@startpbox=\@startpbox
    \let\org@endpbox=\@endpbox
    \let\@startpbox=\ST@astartpbox
    \let\@endpbox=\ST@aendpbox
  \fi
%    \end{macrocode}
%
%    Check if the head of the table should be different for the first
%    and subsequent pages.
%    \begin{macrocode}
  \ifx\@table@first@head\undefined
    \let\@@tablehead=\@tablehead
  \else
    \let\@@tablehead=\@table@first@head
  \fi
%    \end{macrocode}
%    The first part of a supertabular may be moved to the next page
%    if it doesn't fit on the current page. Subsequent parts can not
%    be moved; therefore we will have to switch the definition of
%    |\ST@skippart| around.
%    \begin{macrocode}
  \let\ST@skippage\ST@skipfirstpart
%    \end{macrocode}
%    Now we can estimate the average line height and the height
%    of the first page of the supertabular.
%    \begin{macrocode}
  \estimate@lineht
  \@calfirstpageht
%    \end{macrocode}
% \emph{Extension:} Call the macro to initialize the extension code for 
%  this table.
%    \begin{macrocode}
  \PWSTinit
%    \end{macrocode}
% \emph{Extension:} At this point I know, and have adjusted for, the
% page on which the first part of the table will be printed. It should now
% be safe to print the top caption, if any. Unfortunately, in spite of
% everthing, the \TeX{} page breaking
% mechanism might still think that there is too little space left.
%    \begin{macrocode}
  \if@topcaption \@process@tablecaption \fi
  \noindent
%    \end{macrocode}
%  \emph{Extension:} Finally, subtract the space required by the header
% and the tail (as these don't update the available space when output).
% \changes{v2.3}{2000/04/09}{Subtracted head/tail space in \cs{x@supertabular} }
%    \begin{macrocode}
  \global\advance\ST@pageleft -\ST@headht%
  \ST@trace\tw@{Available space after accounting for header: \the\ST@pageleft}%
  \global\advance\ST@pageleft -\ST@tailht%
  \ST@trace\tw@{Available space after accounting for tail: \the\ST@pageleft}%
}

%    \end{macrocode}
% \end{macro}
%
% \begin{macro}{\PWSTinit}
% \emph{Extension:} This routine initialises the extension data.
% \changes{v2}{1997/11/30}{Added PWSTinit routine}
%    \begin{macrocode}
\newcommand{\PWSTinit}{%
%    \end{macrocode}
%    At the end of processing each supertabular (see later) the number of
% pages consumed by the supertabular is written to the \file{.aux} file.
% At the start of a supertabular, after incrementing the number of 
% supertabulars processed, the prior number of pages are read from
% the file. These are stored in \Lcount{PWSTlastpage}.
%    \begin{macrocode}
  \global\advance\c@PWSTtable\@ne
  \global\expandafter\let\expandafter\PWSTtempc
    \csname PWST@\romannumeral\c@PWSTtable\endcsname
%    \end{macrocode}
%    I have to take account of the fact that there might be no entry in
% the \file{.aux} file, and hence the lastpage number might not be set.
%    \begin{macrocode}
  \ifx\PWSTtempc\relax
    \ST@trace\tw@{Table \the\c@PWSTtable: Processing for the first time}%
    \PWSTlastpage=\@m % = 1000
  \else
    \PWSTlastpage=\PWSTtempc
  \fi
  \ST@trace\tw@{Table \the\c@PWSTtable: last page set to \the\PWSTlastpage}%
%    \end{macrocode}    
%    Set the current page counter to unity.
%    \begin{macrocode}
  \PWSTcurpage=\@ne
%    \end{macrocode}
%    Perform the calculations for the empty table data.
%    \begin{macrocode}
  \PWSTcalchtlines
%    \end{macrocode}
%    Initialise the line counter and set |firstcall| to TRUE.
%    \begin{macrocode}
  \global\PWSTlines=\z@
  \global\firstcalltrue
%    \end{macrocode}
%    If we have the \file{iso} class, then I have to flag that we are
% in a `float'.
% \changes{v2}{1997/12/02}{Added setting of \cs{ifinfloat}}
%    \begin{macrocode}
    \infloattrue
}

%    \end{macrocode}
% \end{macro}
%
% \begin{macro}{\xtabular}
%    We start by looking for an optional argument, which will be duly
%    ignored as it seems to make no sense to try to align a multipage
%    table in the middle...
%
% \emph{Extension:} Use \Lenv{xtabular} instead of \Lenv{supertabular},
% and similarly for the others, so this will
%  not be mentioned explicitly again.
%    \begin{macrocode}
\def\xtabular{%
  \@ifnextchar[{\@supertabular}%]
               {\@supertabular[]}}
%    \end{macrocode}
% \end{macro}
%
% \begin{macro}{\@supertabular}
%    We can now save the preamble of the tabular in a macro.
%    \begin{macrocode}
\def\@supertabular[#1]#2{%
  \def\ST@tableformat{#2}
  \ST@trace\tw@{Starting a new xtabular}%
%    \end{macrocode}
%    Then remember that this is not a \textsf{supertabular*}
%    environment.
%    \begin{macrocode}
  \global\ST@starfalse
%    \end{macrocode}
%    Don't use minipages.
%    \begin{macrocode}
  \global\ST@mpfalse
%    \end{macrocode}
%    Most of the following code is shared between the
%    \Lenv{supertabbular} and \Lenv{supertabular*} environments. So to
%    avoid duplication it is stored in a macro.
%    \begin{macrocode}
  \x@supertabular
%    \end{macrocode}
%    Finally start a normal \Lenv{tabular} environment.
%    \begin{macrocode}
  \expandafter\org@tabular\expandafter{\ST@tableformat}%
  \@@tablehead}

%    \end{macrocode}
% \end{macro}
%
% \begin{macro}{\xtabular*}
%    We start by looking for the optional argument of the tabular
%    environment. 
%    \begin{macrocode}
\@namedef{xtabular*}#1{%
  \@ifnextchar[{\@nameuse{@supertabular*}{#1}}%
               {\@nameuse{@supertabular*}{#1}[]}%]
  }
%    \end{macrocode}
%    We start by saving the intended width and the preamble of the
%    \Lenv{tabular*}.
%    \begin{macrocode}
\@namedef{@supertabular*}#1[#2]#3{%
  \ST@trace\tw@{Starting a new xtabular*}%
  \def\ST@tableformat{#3}%
  \ST@wd=#1\relax
  \global\ST@startrue
  \global\ST@mpfalse
%    \end{macrocode}
%    Now we can call the common code for both environments.
%    \begin{macrocode}
  \x@supertabular
%    \end{macrocode}
%    And we can start a normal \Lenv{tabular*} environment.
%    \begin{macrocode}
  \expandafter\csname org@tabular*\expandafter\endcsname
  \expandafter{\expandafter\ST@wd\expandafter}%
  \expandafter{\ST@tableformat}%
  \@@tablehead}

%    \end{macrocode}
% \end{macro}
%
% \begin{macro}{\mpxtabular}
%    This version of the supertabular environment puts each tabular into
%    a minipage, thus making footnotes possible. We start by looking for
%    an optional argument, which will be ignored as it makes no sense
%    to try and align a multipage table in the middle\ldots
% \changes{v2}{1997/11/30}{Added the mpsupertabular}
%    \begin{macrocode}
\def\mpxtabular{%
  \@ifnextchar[{\@mpsupertabular}%]
               {\@mpsupertabular[]}}
%    \end{macrocode}
%
%    We can now save the preamble in a macro.
%    \begin{macrocode}
\def\@mpsupertabular[#1]#2{%
  \def\ST@tableformat{#2}%
  \ST@trace\tw@{Starting a new mpxtabular}%
%    \end{macrocode}
%    Remember that this is not a \Lenv{mpsupertabular*} environment
%    and also note we have to close the minipage later.
%    \begin{macrocode}
  \global\ST@starfalse
  \global\ST@mptrue
%    \end{macrocode}
%   Since we are about to start a minipage of |\columnwidth| the
%   horizontal alignment will not work. We have to remember the values
%   and then restore them inside the minipage.
%    \begin{macrocode}
  \ST@rightskip \rightskip
  \ST@leftskip \leftskip
  \ST@parfillskip \parfillskip
%    \end{macrocode}
%    Call the code that is common to all the environments.
%    \begin{macrocode}
  \x@supertabular
%    \end{macrocode}
%    Finally, start a normal \Lenv{tabular}
%    \begin{macrocode}
  \minipage{\columnwidth}%
  \parfillskip\ST@parfillskip
  \rightskip \ST@rightskip
  \leftskip \ST@leftskip
  \noindent\expandafter\org@tabular\expandafter{\ST@tableformat}%
  \@@tablehead}

%    \end{macrocode}
% \end{macro}
%
% \begin{macro}{\mpxtabular*}
%    We start by looking for the optional argument of the tabular environment.
%    \begin{macrocode}
\@namedef{mpxtabular*}#1{%
  \@ifnextchar[{\@nameuse{@mpsupertabular*}{#1}}%
               {\@nameuse{@mpsupertabular*}{#1}[]}%]
}
%    \end{macrocode}
%
%    Now we can save the intended width and the preamble of 
%    the \Lenv{tabular*}.
%    \begin{macrocode}
\@namedef{@mpsupertabular*}#1[#2]#3{%
  \ST@trace\tw@{Starting a new mpxtabular*}%
  \def\ST@tableformat{#3}%
  \ST@wd=#1\relax
  \global\ST@startrue
  \global\ST@mptrue
  \ST@rightskip \rightskip
  \ST@leftskip \leftskip
  \ST@parfillskip \parfillskip
%    \end{macrocode}
%    Now is the time to call the common code for both environments.
%    \begin{macrocode}
  \x@supertabular
%    \end{macrocode}
%    And we can start a normal \Lenv{tabular*} environment.
%    \begin{macrocode}
  \minipage{\columnwidth}%
  \parfillskip\ST@parfillskip
  \rightskip \ST@rightskip
  \leftskip \ST@leftskip
  \noindent\expandafter\csname org@tabular*\expandafter\endcsname
  \expandafter{\expandafter\ST@wd\expandafter}%
  \expandafter{\ST@tableformat}%
  \@@tablehead}%

%    \end{macrocode}
% \end{macro}
%
% \begin{macro}{\endxtabular}
% \begin{macro}{\endxtabular*}
%    These close the \Lenv{xtabular} and \Lenv{xtabular*}
%    environments.
%
%    For the extension, this macro has been modified to write out 
% to the \file{.aux} file the 
% number of pages used for the supertabular. 
%    \begin{macrocode}
\def\endxtabular{%
  \ifx\@table@last@tail\undefined
    \@tabletail
  \else
    \@table@last@tail
  \fi
  \csname endtabular\ifST@star*\fi\endcsname
%    \end{macrocode}
%   While studying the original code to determine where additions were
% needed for the extension, I realized that the last part of
% the |\end...| code was common to all the environments. I have broken
% it out into a seperate routine which also includes the modification
% needed for the extension.
%    \begin{macrocode}
  \x@endsupertabular
%    \end{macrocode}
%    And back to the original code.
%    \begin{macrocode}
  \ST@trace\tw@{Ended a xtabular\ifST@star*\fi}%
  }
%    \end{macrocode}
%
%    The definition of the ending of the \Lenv{xtabular*}
%    environment is simple:
%    \begin{macrocode}
\expandafter\let\csname endxtabular*\endcsname\endxtabular
%    \end{macrocode}
% \end{macro}
% \end{macro}
%
% \begin{macro}{\x@endsupertabular}
%     This macro contains the code that is common to all the |\end...|
% commands. It includes the modification required for the extension.
% \changes{v2}{1997/11/30}{Added x@endsupertabular routine}
%    \begin{macrocode}
\newcommand{\x@endsupertabular}{%
%    \end{macrocode}
%    Restore the original definition of |\@tabularcr|
%    \begin{macrocode}
  \ST@restore
%    \end{macrocode}
%    Check if we have to insert a caption and restore to default
%    behaviour of putting captions at the top.
%    \begin{macrocode}
  \if@topcaption
  \else
    \@process@tablecaption
    \global\@topcaptiontrue
  \fi
%    \end{macrocode}
%    Restore the meaning of |\\| to the one it had before the start
% of this environment. Also re-initialize some control-sequences
%    \begin{macrocode}
  \global\let\\=\@oldcr
  \global\let\@table@first@head\undefined
%%%  \global\let\@table@last@tail\undefined
  \global\let\@process@tablecaption\relax
%    \end{macrocode}
% \emph{Extension:} For the extension, write the number of the last page 
% to the \file{.aux} file. Also, if we are in the \file{iso} class, reset
% the `float' flag.
% \changes{v2}{1997/12/02}{Added resetting of \cs{ifinfloat}}
%    \begin{macrocode}
  \PWSToplastpagenum
    \infloatfalse
}

%    \end{macrocode}
% \end{macro}
%
% \begin{macro}{\PWSToplastpagenum}
%  \emph{Extension:} This routine is responsible for writing the number of the
% last page of the supertabular to the \file{.aux} file.
%
% What gets written is
% |\PWST@vi{4}|, assuming that the value of \Lcount{c@PWSTtable} is 6
% and the value of \Lcount{PWSTcurpage} is 4.
% \changes{v2}{1997/11/30}{Added PWSToplastpagenum routine}
% \changes{v2.3}{2000/04/09}{Just checked for number of entry lines in \cs{PWSToplastpagenum} }
%    \begin{macrocode}
\newcommand{\PWSToplastpagenum}{%
%    \end{macrocode}
%    There are a number of cases to consider. The first decision is whether
% the current page is the previously calculated last page.
%    \begin{macrocode}
  \ifnum\PWSTcurpage=\PWSTlastpage
%    \end{macrocode}
%     The current table ends on the calculated last page. There are four
% cases to consider:
% \begin{enumerate}
% \item The table has not overflowed (|firstcall| is TRUE) and the table
%       is not empty --- this page is still the last page.
% \item The table has not overflowed (|firstcall| is TRUE) and the table
%       is empty --- this page is after the actual last page, so decrease
%       the page number.
% \item The table has overflowed (|firstcall| is FALSE) and the overflow
%       is large enough to generate a non-empty table on the next page ---
%       increment the page number.
% \item The table has overflowed (|firstcall| is FALSE) and the overflow
%       is small enough to generate an empty table on the next page ---
%       this page is still the last page.
% \end{enumerate}
%    \begin{macrocode}
    \iffirstcall  % on last, no overflow
%%      \ifnum\PWSTlines < \PWSTlasthead % this table is empty
      \ifnum\PWSTlines < \@ne            % this table is empty
        \global\advance\PWSTcurpage \m@ne
      \fi
    \else % overflow
%%      \ifnum\PWSTlines > \tw@ % enough for another page
      \ifnum\PWSTlines > \z@    % enough for another page
        \global\advance\PWSTcurpage \@ne
      \fi
    \fi
  \else
%    \end{macrocode}
%    The table has ended on a page that is not the calculated last page.
% If the table is empty, then decrement the page number, else this is
% the last page.
%    \begin{macrocode}
%%    \ifnum\PWSTlines < \PWSThead % empty table
    \ifnum\PWSTlines < \@ne        % empty table
      \global\advance\PWSTcurpage \m@ne
    \fi
  \fi
%    \end{macrocode}
%    Finally, write out the `new' last page number.
%    \begin{macrocode}
  \if@filesw\immediate\write\@auxout%
    {\gdef\string\PWST@\romannumeral\c@PWSTtable{\the\PWSTcurpage}}
    \ST@trace\tw@{Table \the\c@PWSTtable:\MessageBreak
                  wrote \the\PWSTcurpage\space as the last page}%
  \fi
}

%    \end{macrocode}
% \end{macro}
%
%
% \begin{macro}{\endmpxtabular}
% \begin{macro}{\endmpxtabular*}
%    These close the \Lenv{mpxtabular} and \Lenv{mpxtabular*}
%    environments.
%    \begin{macrocode}
\def\endmpxtabular{%
  \ifx\@table@last@tail\undefined
    \@tabletail
  \else
    \@table@last@tail
  \fi
  \csname endtabular\ifST@star*\fi\endcsname
  \endminipage
%    \end{macrocode}
%    Now call the common code for all |\end...|.
%    \begin{macrocode}
  \x@endsupertabular
%    \end{macrocode}
% Finish per the original code.
%    \begin{macrocode}
  \ST@trace\tw@{Ended an mpxtabular\ifST@star*\fi}%
  }
%    \end{macrocode}
% 
%    The definition of the ending of the \Lenv{mpxtabular*}
%    environment is simple:
%    \begin{macrocode}
\expandafter\let\csname endmpxtabular*\endcsname\endmpxtabular

%    \end{macrocode}
% \end{macro}
% \end{macro}
%
% \begin{macro}{\ST@restore}
%    This macro restores the original definitions of the macros that
%    handle parbox entries and the `end of row' macros.
%    \begin{macrocode}
\def\ST@restore{%
  \ifx\undefined\@classix
    \let\@tabularcr\org@tabularcr
  \else
    \let\@arraycr\org@tabularcr
  \fi
  \let\@startpbox\org@startpbox
  \let\@endpbox\org@endpbox
}

%    \end{macrocode}
% \end{macro}
%
% \begin{macro}{\inner@tabular}
% \begin{macro}{\inner@tabular*}
%    In order to facilitate complete \Lenv{tabular} environments to be in
%    a cell of a \Lenv{supertabular} we need to adapt the definition
%    of the original environments. For the inner \Lenv{tabular} a
%    number of definitions have to be restored.
%    \begin{macrocode}
\def\inner@tabular{%
  \ST@restore
  \let\\=\@oldcr
  \noindent
  \org@tabular}
\@namedef{inner@tabular*}{%
  \ST@restore
  \let\\=\@oldcr
  \noindent
  \csname org@tabular*\endcsname}

%    \end{macrocode}
% \end{macro}
% \end{macro}
%
% \begin{macro}{\ST@cr}
%    This macro is called by each |\\| inside the tabular environment.
%    It updates the estimate of the amount of space left on the
%    current page and starts a new page if necessary.
%    \begin{macrocode}
\def\ST@cr{%
  \noalign{%
    \ST@trace\thr@@{Parbox height: \the\ST@pboxht\MessageBreak 
                    Line height: \the\ST@lineht}%
    \ifnum\ST@pboxht<\ST@lineht
%    \end{macrocode}
%    If there is a non-empty line, but an empty parbox, then
%    |\ST@pboxht| might be non-zero, but too small thereby breaking
%    the algorithm. Therefore we estimate the height of the line to be
%    |\ST@lineht| in this case, and store it in |\ST@prevht|.
%    \begin{macrocode}
      \global\advance\ST@pageleft -\ST@lineht
      \global\ST@prevht\ST@lineht
    \else
%    \end{macrocode}
%    When the parbox is not empty we take its height into account
%    plus a little extra.
% \changes{v2.3}{2000/04/09}{Used \cs{PWST@xentrystretch} in \cs{ST@cr} }
%    \begin{macrocode}
      \global\advance\ST@pboxht \PWST@xentrystretch\ST@pboxht
      \global\advance\ST@pboxht \ST@stretchht
      \ST@trace\thr@@{Added par box with height \the\ST@pboxht}%
      \global\advance\ST@pageleft -\ST@pboxht
      \global\ST@prevht\ST@pboxht
      \global\ST@pboxht\z@
    \fi
%    \end{macrocode}
%    |\ST@toadd| is the value of the optional argument of |\\|.
%    \begin{macrocode}
     \global\advance\ST@pageleft -\ST@toadd
     \global\ST@toadd=\z@
     \ST@trace\thr@@{Space left for xtabular: \the\ST@pageleft}%
%    \end{macrocode}
% \emph{Extension:} Increment the line number at this point.
%    \begin{macrocode}
     \global\advance\PWSTlines \@ne
     \ST@trace\thr@@{Line counter incremented by one to: \the\PWSTlines}%
  } % end of noalign
%    \end{macrocode}
%    In general, when the |\ST@pageleft| has become negative, 
%    the last row was so high
%    that the supertabular doesn't fit on the current page. In this case we
%    skip the current page and start at the top of the next one; otherwise
%    \TeX{} will move this part of the table to a new page anyway, probably
%    with a message about an overfull |\vbox|.
%
% \emph{Extension:} For the extension I do some special handling if we are on the last
% page. Essentially the idea is not to start a new page, but to continue
% on the current page, noting any overflow.
%    \begin{macrocode}
  \ifnum\PWSTcurpage=\PWSTlastpage
    \PWST@lastpagecr
  \else
%    \end{macrocode}
%    Execute the original code.
%    \begin{macrocode}
    \ifnum\ST@pageleft<\z@
      \ST@skippage
    \else
%    \end{macrocode}
%    When there is not enough space left on the current page, we start 
%    a new page. To compute the amount of space needed we use the height
%    of the previous line (|\ST@prevht|) as an estimate of the height
%    of the next line. If we are processing an \Lenv{mpsupertabular}
%    we also need to take the height of the footnotes into account.
%    \begin{macrocode}
      \noalign{\global\@tempdima\ST@tailht
        \global\advance\@tempdima\ST@prevht
      \ifST@mp
        \ifvoid\@mpfootins\else
          \global\advance\@tempdima\ht\@mpfootins
          \global\advance\@tempdima 3pt
        \fi
      \fi} % end noalign
      \ifnum\ST@pageleft<\@tempdima
        \ST@newpage
      \else
%    \end{macrocode}
%    This line is necessary because the tablehead has to be inserted
%    \emph{after} the |\if\else\fi|-clause. For this purpose |\ST@next| is
%    used. In the middle of tableprocessing it should be an \emph{empty}
%    macro (\emph{not} |\relax|).
%    \begin{macrocode}
        \noalign{\global\let\ST@next\@empty}%
      \fi
    \fi
%    \end{macrocode}
% \emph{Extension:} Close off the |\iflastpage|;
%    \begin{macrocode}
  \fi
%    \end{macrocode}
%    and finish per the original code.
%    \begin{macrocode}
  \ST@next}

%    \end{macrocode}
% \end{macro}
%
% \begin{macro}{\PWST@lastpagecr}
% \emph{Extension:} This routine handles newlines on the last page
% of a supertabular. The idea is that when we are on the last page the
% table continues to be processed until the end without calling for a 
% newpage even if the table will be too long. I do need to record whether
% or not the table has `overflowed' the allowable space on the page.
% The code is very similar to the last part of the code for |\ST@cr|.
% \changes{v2}{1997/11/30}{Added PWST@lastpagecr routine}
%    \begin{macrocode}
\newcommand{\PWST@lastpagecr}{%
  \noalign{%
    \ifnum\ST@pageleft<\z@
%    \end{macrocode}
%    The table has overflowed, so record the fact.
%    \begin{macrocode}
      \PWST@setfirstcall
    \fi
%    \end{macrocode}
%    Now continue along the lines of |\ST@cr|.
%    \begin{macrocode}
    \global\@tempdima\ST@tailht
    \global\advance\@tempdima\ST@prevht
      \ifST@mp
        \ifvoid\@mpfootins\else
          \global\advance\@tempdima\ht\@mpfootins
          \global\advance\@tempdima 3pt
        \fi
      \fi
      \ifnum\ST@pageleft<\@tempdima
%    \end{macrocode}
%    Again, the table has overflowed.
%    \begin{macrocode}
        \PWST@setfirstcall
      \fi
%    \end{macrocode}
%    Finish like |\ST@cr|.
%    \begin{macrocode}
      \global\let\ST@next\@empty
      } % end noalign
}

%    \end{macrocode}
% \end{macro}
%
% \begin{macro}{\PWST@setfirstcall}
% \emph{Extension:} This routine records that a table on the last page 
% has overflowed
% by setting |firstcall| to FALSE. If it is the first such overflow
% it also zeroes the line counter.
% \changes{v2}{1997/11/30}{Added PWST@setfirstcall routine}
%    \begin{macrocode}
\newcommand{\PWST@setfirstcall}{%
  \iffirstcall
    \global\firstcallfalse
    \global\PWSTlines=\z@
    \ST@trace\thr@@{Overflow on last page. Line counter set to \the\PWSTlines}%
  \fi
}

%    \end{macrocode}
% \end{macro}
%
% \begin{macro}{\ST@skipfirstpart}
%    This macro skips the current page and moves the entire supertabular
%    that has been built so far to the next page.
%    \begin{macrocode}
\def\ST@skipfirstpart{%
  \noalign{%
    \ST@trace\tw@{Tabular too high, moving to next page}%
%    \end{macrocode}
%    In order for this to work properly we need to adapt the value of
%    |\ST@pageleft|. When this macro is called it has a negative value.
%    We should add the height of the next page to that (|\@colroom|).
%    From the result the `normal' height of the supertabular should be
%    subtracted (|\@colroom - \pagetotal|). This could be coded as follows:
%    \begin{verbatim}
%      \ST@dimen\@colroom
%      \advance\ST@dimen-\pagetotal
%      \global\advance\ST@pageleft\@colroom
%      \global\advance\ST@pageleft-\ST@dimen
%    \end{verbatim}
%    However, note that |\@colroom| is added \emph{and} subtracted. Thus the
%    code can be simplified to:
%    \begin{macrocode}
    \global\advance\ST@pageleft\pagetotal
%    \end{macrocode}
%    Then we can set |\ST@pagesofar| to zero and start the new page.
%    \begin{macrocode}
    \global\ST@pagesofar\z@
    \newpage
%    \end{macrocode}
%    Finally we make sure that this macro can only be executed once for
%    each supertabular by changing the definition of |\ST@skippage|.
%    \begin{macrocode}
    \global\let\ST@skippage\ST@newpage
  }
}

%    \end{macrocode}
% \end{macro}
%
% \begin{macro}{\ST@newpage}
%    This macro performs the actions necessary to start a new page.
%
%    This macro is also modified for the extension to supertabluar.
%
%    \begin{macrocode}
\def\ST@newpage{%
  \noalign{\ST@trace\tw@{Starting new page, writing tail}}%
%    \end{macrocode}
%    Output |\tabletail|, close the tabular environment, close a minipage
%    if necessary, output all material and start a fresh new page.
%    \begin{macrocode}
  \@tabletail
  \ifST@star
    \csname endtabular*\endcsname
  \else
    \endtabular
  \fi
  \ifST@mp
    \endminipage
  \fi
%    \end{macrocode}
%    Then we make sure that |\ST@skippage| can no longer be executed for
%    this supertabular by changing its definition.
%    \begin{macrocode}
  \global\let\ST@skippage\ST@newpage
%    \end{macrocode}
% On with the output.
%
% \emph{Extension:} The original code had the next line as 
% |\newpage\@calnextpageht|. However, if the general header has a
% vertical height that differs from the first header, then the table
% on the continuation pages may run short or, more disconcerting, long.
% The extension, I think, cures that by using a different algorithm to
% calculate the height on the next page.
% \changes{v2}{1997/11/30}{Replaced @calnextpageht by PWSTcalnextpageht in ST@newpage}
%    \begin{macrocode}
  \newpage\PWSTcalnextpageht
  \ST@trace\tw@{writing head}%
%    \end{macrocode}
% \emph{Extension:} The original code just let |\ST@next| to |\@tablehead|. 
% The extension
% has to handle the special case of of the heading on the last page.
%    \begin{macrocode}
  \PWSTsethead
%    \end{macrocode}
%    Now we are back to the original supertabular code.
%    \begin{macrocode}
  \ifST@mp
    \noindent\minipage{\columnwidth}%
    \parfillskip\ST@parfillskip
    \rightskip \ST@rightskip
    \leftskip \ST@leftskip
  \fi
  \noindent
  \ifST@star
    \expandafter\csname org@tabular*\expandafter\endcsname
    \expandafter{\expandafter\ST@wd\expandafter}%
    \expandafter{\ST@tableformat}%
  \else
    \expandafter\org@tabular\expandafter{\ST@tableformat}%
  \fi}

%    \end{macrocode}
% \end{macro}
%
% \begin{macro}{\PWSTsethead}
% \emph{Extension:} This is more extension code for use within |\ST@newpage|.
% It provides the proper table head for the page about to be processed.
% \changes{v2}{1997/11/30}{Added PWSTsethead routine}
%    \begin{macrocode}
\newcommand{\PWSTsethead}{%
%    \end{macrocode}
% First the line counter is zeroed.
%    \begin{macrocode}
  \global\PWSTlines=\z@
  \ST@trace\thr@@{Newpage, line counter set to: \the\PWSTlines}%
%    \end{macrocode}
% The current page counter is incremented
% and it is checked against the old page counter to see if this is
% the last page of this supertabular. 
%    \begin{macrocode}
  \global\advance\PWSTcurpage\@ne
  \ST@trace\tw@{Table \the\c@PWSTtable:\MessageBreak
                current page = \the\PWSTcurpage,\MessageBreak
                last page = \the\PWSTlastpage}%
  \ifnum\PWSTcurpage=\PWSTlastpage
    \ST@trace\tw@{Newpage is the last page}%
%    \end{macrocode}
% We are on the last page. If there are more than
% one pages and the last table heading has been specified,
% then the heading is set to |\@table@last@head|, otherwise it is set
% to |\@tablehead|.
%    \begin{macrocode}
    \ifnum\PWSTcurpage>\@ne
      \ifx\@table@last@head\relax
        \let\ST@next\@tablehead
        \ST@trace\tw@{Set heading to tablehead}%
      \else
        \let\ST@next\@table@last@head
        \ST@trace\tw@{Set heading to tablelasthead}%
      \fi
    \fi
  \else
%    \end{macrocode}
%    We are not on the last page, so just set the heading to |\@tablehead|.
%    \begin{macrocode}
    \let\ST@next\@tablehead
    \ST@trace\tw@{Set heading to tablehead}%
  \fi
}

%    \end{macrocode}
% \end{macro}

%    The end of this package
%    \begin{macrocode}
%</xtab>
%    \end{macrocode}
%
%
%
% \bibliographystyle{alpha}
%
% \begin{thebibliography}{GMS94}
%
% \bibitem[GMS94]{GOOSSENS94}
% Michel Goossens, Frank Mittelbach, and Alexander Samarin.
% \newblock {\em The LaTeX Companion}.
% \newblock Addison-Wesley Publishing Company, 1994.
%
% \bibitem[Wil96]{PRW96i}
% Peter~R. Wilson.
% \newblock {\em {LaTeX for standards: The LaTeX package files user manual}}.
% \newblock NIST Report NISTIR, June 1996.
%
% \end{thebibliography}
%
%
% \Finale
% \PrintIndex
%
\endinput

%% \CharacterTable
%%  {Upper-case    \A\B\C\D\E\F\G\H\I\J\K\L\M\N\O\P\Q\R\S\T\U\V\W\X\Y\Z
%%   Lower-case    \a\b\c\d\e\f\g\h\i\j\k\l\m\n\o\p\q\r\s\t\u\v\w\x\y\z
%%   Digits        \0\1\2\3\4\5\6\7\8\9
%%   Exclamation   \!     Double quote  \"     Hash (number) \#
%%   Dollar        \$     Percent       \%     Ampersand     \&
%%   Acute accent  \'     Left paren    \(     Right paren   \)
%%   Asterisk      \*     Plus          \+     Comma         \,
%%   Minus         \-     Point         \.     Solidus       \/
%%   Colon         \:     Semicolon     \;     Less than     \<
%%   Equals        \=     Greater than  \>     Question mark \?
%%   Commercial at \@     Left bracket  \[     Backslash     \\
%%   Right bracket \]     Circumflex    \^     Underscore    \_
%%   Grave accent  \`     Left brace    \{     Vertical bar  \|
%%   Right brace   \}     Tilde         \~}
